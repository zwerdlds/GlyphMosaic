\section{System Overview}
\subsection{Scope}
GlyphMosaic is a specialized graphic design application.
It comprises the following functionality:
\begin{itemize}
  \item Enable the user to finely specify the parameters with which the output image is generated.
  \item Produce the output image.
\end{itemize}


\subsection{Context}
The application is installed on a host system.
Users interact with the system's GUI to create, load, and modify graphical mosaics.

\sidiagram
{diagrams/context_diagram}
{A hypothetical local install of GlyphMosaic and related components.}
{fig:contextDiagram}
{\diagsize}


\subsection{Audience}
This document is intended for contributors and consumers involved in the development, maintenance, and use of the GlyphMosaic application.
It aims to provide audiences with a complete description of the underlying organization of the application, facilitating insight into how all system components interact to deliver system functions.
By reading this document developers can make informed decisions about how to design and implement functionalities that will deliver an optimal user experience.
This document's purpose is to be a comprehensive overview of the GlyphMosaic software architecture to support all audience members involved.


\subsection{Statement of Purpose}
GlyphMosaic is described in this architectural document.
Diagrams are also supplied to facilitate comprehension of the system internals.
All aspects of the GlyphMosaic architecture, including context, modules, and code structure are described.
End-user-facing functionality is also described.
This document also highlights architectural compromises and alternatives which were considered during engineering phases of the system.
This document should act as a reference for any person interested in better understanding the GlyphMosaic architecture.

This report attempts to accomplish the following goals:
\begin{itemize}
  \item Describe the complete architecture of the system.
  \item Describe available process alternative methods and their trade-offs.
  \item Provide a broad statement of use for the system itself.
\end{itemize}