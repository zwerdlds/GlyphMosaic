\section{Requirements}
Requirements for GlyphMosaic aim to create a responsive and productive environment for its users.
The GlyphMosaic System provides a specialized method to build unique graphic mosaic designs.


\subsection{Functionality}
GlyphMosaic’s main built-in functionality is focused entirely on the workflow of creating mosaics.
In the most common workflow, a user will load source image and source text into the program, adjust various settings and bitmaps to their needs, render a high-resolution version of the output mosaic, and then save that file for use outside the application.

\subsection{Differentiation}
Existing systems perform a similar, but limited subset, of functionality of the system.
Textaizer\cite{textaizer}, for example, has the ability to specify various line patterns, such as spirals.
Many other examples exist that focus solely on LTR horizontal mosaics.

GlyphMosaic does not have these limitations.
The creator is able to specify the glyph line pattern by specifying a region line kernel for each region.
The system then determines a text path which encompasses the entire region, and develops a

\subsection{Use Cases}
\label{sec:use_cases}

\begin{itemize}
  \lbodyitem{System Installation}{
  \item Users should be able to download, install, and run the system on their own devices.
        }
        \lbodyitem{Content Creation}{
  \item Users should be able to create and modify GM documents.
        }
\end{itemize}

\sidiagram
{diagrams/use_case_diagram}
{GlyphMosaic models a specialized graphic design workflow.}
{fig:useCaseDiagram}
{\diagsize}