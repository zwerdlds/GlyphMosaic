\section{Requirements}
Requirements for GlyphMosaic aim to create a responsive and productive environment for its users.
The GlyphMosaic System provides a specialized method to build unique graphic mosaic designs.


\subsection{Functionality}
GlyphMosaic’s main built-in functionality is focused entirely on the workflow of creating mosaics.
In the most common workflow,
\begin{enumerate}
      \item a user will load the source image and text into the program...
      \item ...adjust various document settings to their needs...
      \item ...render a high-resolution version of the output mosaic...
      \item ...save that file for use outside the application.
\end{enumerate}


\subsection{Differentiation}
Existing systems perform a similar, but limited subset, of the functionality of the system.
One system can specify various line patterns, such as spirals.
Many other examples exist that focus solely on LTR horizontal mosaics.

GlyphMosaic's pathing is \gls{bitmap}-based, using \glspl{glyph_path_kernel} instead.
Mosaic creators can specify the glyph line pattern by specifying a region line kernel for each region.
The system then determines a text path that encompasses the entire region, and stamps glyphs from the \gls{src_txt} along the \gls{glyph_path}, scaling according to the \gls{density} from the \gls{src_img} to arrive at the \gls{out_image}.

Because glyph paths are built from bitmaps instead of mathematical expressions, designers have more options to specify complexity.
This path complexity enables more interesting patterns, such as following a fold in a piece of fabric, or the curves on a bowl.

\subsection{Use Cases}
\label{sec:use_cases}

\begin{itemize}
      \lbodyitem{System Installation}{
      \item Users must be able to download, install, and run the system on their own devices.
            }
            \lbodyitem{Content Creation}{
      \item Users must be able to create and modify GM documents.
            }
\end{itemize}

This workflow is further detailed in \prettyref{sec:user_workflow_model}.