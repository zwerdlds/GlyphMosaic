\subsection{Point}

\lidiagram{diagrams/subsystems/point/abstract}
{The data that are used and produced by the point subsystem.}
{fig:pt_abstract}

The Point subsystem is responsible for determining the location and rotation of image glyphs.
It uses Lines data generated by the Line subsystem, along with the mask, kerning, and rotation sample size.

To do this, the system walks a region's lines, skipping a number of points corresponding to the kerning parameter.
These points make up a candidate set of points which is then intersected with the mask, giving the valid points for the region.

These values are then used by Letters to determine the location to sample in the source image.


\slidiagram{diagrams/subsystems/point/inter}
{The point subsystem depends on types in Line and is used by Letter.}
{fig:lin_inter}
{.5}

\subsubsection{Mechanics}
Point operates per-region.
Clients are responsible for setting up a new PointManager for each region.
That manager must be used to wire up the Lines subscriber to the output from the line subsystem, and the publisher to the Letter Points subscriber.

\slidiagram{diagrams/subsystems/point/intra}
{The point subsystem types do not depend on any other system types.  However, it is necessary for other systems to be wired up at runtime.}
{fig:lin_inter}
{.5}

\subsubsection{Usage}
