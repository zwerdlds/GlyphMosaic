\subsection{Persistence}
\permod\ is responsible for reading and writing system parameters to and from self-contained documents.
This enables the system to recreate the in-progress work of a user after the application has been closed.

Events that trigger a document write or load are triggered by the user via the interface.
Load events propagate document parameters associated with them to every subsystem.
To eliminate coupling, these relationships will be constructed by the system at runtime, such that this subsystem is fully distinct from others.

\slidiagram{diagrams/subsystems/persistence/inter}
{Persistence inter-system dependencies.}
{fig:pers_inter}
{.5}


\subsubsection{Mechanics}
\perftype\ acts as a subscriber to property change events.
It implements a listener for all the associated fields which a user can modify.

To ensure correct dataflows, an enum provides typed data fields.
\permod\ then uses that enum to generate a visitor that acts as a single entry point for clients.


\lidiagram{diagrams/subsystems/persistence/intra}
{Persistence subsystem components.}
{fig:pers_intra}
{}

The subsystem uses a SQL-style transactional approach to writing data to a file:
\begin{itemize}
      \item On startup, an UnQLite document is created and a new transaction is initiated.
      \item When \verb|gm::persistence::PersistenceWorker::save()| is invoked, the UnQLite transaction is committed and a new one is begun.
      \item When \verb|...::save_as()| is invoked, the UnQLite document path is set, and \verb|...::save()| is invoked.
      \item When .\verb|..::load()| is invoked, the UnQLite file is opened, replacing the existing one, and all parameters are re-published.
\end{itemize}


\subsubsection{Usage}
In addition to document-related properties, this subsystem contains an agent responsible for initiating the document write-out and read processes.
Agents are registered for the \uimod\ "Save", "Save As", and "Load" events.
In general, the subsystem functions as follows:
\begin{itemize}
      \item Save \\
            Write the current state of the document to a serialized location.
            If the file does not already have an associated location and needs one, this will prompt the user to select a location (TODO: the method with which this is implemented is to be determined).
      \item Save As \\
            Write the current state of the document from a serialized location to a specific location.
      \item Load \\
            Taking a path argument, publish the necessary parameter changes to recreate the deserialized document at that location.
            This propagates all parameters to the \docftype.
\end{itemize}