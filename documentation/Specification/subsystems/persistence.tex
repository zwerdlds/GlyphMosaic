\subsection{Persistence}
\permod\ is responsible for reading and writing parameters to and from system-external.
This enables the system to recreate the work of a user after the application has been terminated.
The system operates on a "document" basis: each document is contained in a (main) application window.
This document can then be saved to a specific location on the user's local storage device.
This is implemented by storing an UnQLite-based copy of the latest parameters propagated by other systems.

Any other subsystem can propagate changes to the document.
In actuality, the \uimod\ is the sole source of changes, due to all user-facing properties being directly editable via that subsystem.
Additional properties, such as Lines, Points, and Letters are not user-mutable.
Therefore, the document can be recreated without them, and are omitted from persistence.
Omitting these properties trades initial load time for document size.

Events that trigger a document write or load are triggered by the user via the interface.
Load events propagate document parameters associated with them to every subsystem.
These are handled via the persistence LoadEventSubscriber, which is supplied with the necessary events to recreate the in-memory state of the document.

\paragraph{Parameter Quantity and Design Implications}
There are numerous parameters.
To reduce system complexity, they are collected into a single interface: PersistenceEventHandler.
This enables the system to process Region creation and deletion, as well as the save location of the document.
To ensure a complete and performant system, the subsystem uses the Roopes visitor implementation to ensure all enums are consumed.

\slidiagram{diagrams/subsystems/persistence/inter}
{Persistence inter-system dependencies.}
{fig:pers_inter}
{.5}


\subsubsection{Mechanics}
\perftype\ acts as a subscriber to property change events.
It implements a listener for all the associated fields which a user can modify.

To ensure correct dataflows, an enum provides typed data fields.
\permod\ then uses that enum to generate a visitor that acts as a single entry point for clients.

The subsystem uses a SQL-style transactional approach to writing data to a file:
\begin{itemize}
      \item On startup, an UnQLite document is created and a new transaction is initiated.
      \item When save is invoked, the UnQLite transaction is committed and a new one is begun.
      \item When save as is invoked, the UnQLite document path is set, and save is invoked.
      \item When load is invoked, the UnQLite file is opened, replacing the existing one, and all parameters are re-published.
\end{itemize}


\lidiagram{diagrams/subsystems/persistence/intra}
{Persistence subsystem components.}
{fig:pers_intra}

\lidiagram{diagrams/subsystems/persistence/events}
{Persistence subsystem events.  This is a continuation from \prettyref{fig:pers_intra}}
{fig:pers_events}


\subsubsection{Usage}
In addition to document-related properties, this subsystem contains an agent responsible for initiating the document write-out and read processes.
Agents are registered for the \uimod\ "Save", "Save As", and "Load" events.
In general, the subsystem functions as follows:
\begin{itemize}
      \item Save \\
            Write the current state of the document to a serialized location.
            If the file does not already have an associated location and needs one, this will prompt the user to select a location (TODO: the method with which this is implemented is to be determined).
      \item Save As \\
            Write the current state of the document from a serialized location to a specific location.
      \item Load \\
            Taking a path argument, publish the necessary parameter changes to recreate the deserialized document at that location.
            This propagates all parameters to the \docftype.
\end{itemize}