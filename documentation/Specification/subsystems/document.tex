\newcommand{\docarea}[3]{
      \index{#1}\pbodyitem{#2}{#3}
}

\newcommand{\extparamref}[3]{
      \index{#1}\pbodyitem{#2}{#3}
}

\subsection{Document}
The document namespace contains logic concerning the state of the persisted document.
This sub-system listens for the application to change parameters of the document, and persists them.
Additionally, when a program document file is loaded, it will send notifications to subscribed components.

\lidiagram{diagrams/subsystems/document}
{Document subsystem components.}
{fig:doc_components}

\subsubsection{Mechanics}
A \docsftype\ is created from a \docsfftype\ by client code in the root of the application.
That \docsftype\ that is passed into users of \docmod\ as a \docftype\ for further use.

Each \docftype\ supposes a single active document at a given time.
I.e: each document window should have a single attached \docftype.

\subsubsection{Usage}
\docftype\ is responsible for registering agents responsible for handling changes to the active document.
The bulk of subsystem-internal implementation includes <<PersistenceWorker>>, which listens for parameters to change, and writes them to a file.

The implementation of PersistenceWorker is bifurcated on target: WASM vs File Storage.

In general, the subsystem functions as follows:
\begin{itemize}
      \item Initialize Document \\
            Publish the necessary parameter changes to recreate the default document.
      \item Save \\
            Write the current state of the document to a serialized location.
      \item Save As \\
            Write the current state of the document to a serialized location to a specific location.
      \item Load Document \\
            Taking a path argument, publish the necessary parameter changes to recreate the deserialized document at that location.
      \item Document Property Change \\
            Perform the given task on the persisted document.
\end{itemize}

