\section{Glossary}
\begin{itemize}
  \pbodyitem{Bitmap}{
    A 2-dimensional matrix of pixels, representing an image.
    Each pixel may contain an arbitrary number of channels.
    In practice, bitmaps typically contain 3 channels (in the case of source images and output images) or a single channel (in the case of region masks and path dilation masks).
  }
  \pbodyitem{Density}{
    Pixels from the source image must be sampled and converted into scaled glyphs.
    This requires the pixel's channels to be interpreted and mapped to, effectively, a font size.
    A pixel's density determines this result.
    Density corresponds to the pixel's apparent brightness.
    This is calculated using the luminosity of the pixel.
  }
  \pbodyitem{Document}{
    A file which contains all the necessary elements to recreate the output image.
    A user will create and modify documents for each output file they wish to create.
    Documents are stored and managed outside this system, by the host.
  }
  \pbodyitem{Glyph}{
    A single textual element.
    Generally corresponds to a single letter, but that definition is incomplete due to complexities introduced by Unicode, which the system utilizes in the process of rendering.
  }
  \pbodyitem{Output Image}{
    A system-rendered bitmap.
    Users may export output images
  }
  \pbodyitem{Path Dilation Masks}{
    The system generates a series of dilations around the given region line kernel.
    This involves taking the prior iteration, or kernel in the case of the first iteration, and accreting (dilating) pixels within a given radius of existing pixels, generating the next iteration.
    In this method, a series of concentric lines can be generated, enabling a text path to be calculated.
  }
  \pbodyitem{Glyph Path}{
    Glyphs are written along a path at regular intervals referred to as the glyph path.
  }
  \pbodyitem{Glyph Path Kernel}{
    The glyph path is generated by drawing concentric circles around a set of pixels.
    This set of pixels is the glyph path kernel.
    Each pixel in the kernel which is active is referred to as an Active Glyph Path Kernel Pixel, or AKP.
  }
  \pbodyitem{Render}{ }
  \pbodyitem{Region}{ }
  \pbodyitem{Region Mask}{ }
  \pbodyitem{Source Image}{
    A bitmap which serves as an oracle for individual pixels.
    Source image pixels are sampled to determine the size and color of each glyph before stamping.
  }
  \pbodyitem{Source Text}{
    A corpus of text is specified as a source for glyphs, which are then applied to the calculated glyph data elements.
    These glyphs are scaled, translated, and rotated accordingly, and then stamped on a bitmap, forming the final image render.
  }
\end{itemize}