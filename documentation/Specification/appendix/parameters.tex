\newcommand{\paritem}[4]{
  \pbodyitem{#1}{
    \index{Parameters!#1}
    \textbf{Domain:} {#3} \newline
    \textbf{Notated:} {#4} \newline
    {#2}
  }
}

\newcommand{\FontParSymbol}{\(F\)}
\newcommand{\SrcImgParSymbol}{\(SI\)}
\newcommand{\SrcTxtParSymbol}{\(ST\)}
\newcommand{\OutSclParSymbol}{\(OS\)}
\newcommand{\RegMskParSymbol}{\(M_r\)}
\newcommand{\GlyphPathKernelParSymbol}{\(PK_r\)}
\newcommand{\GtrHtParSymbol}{\(GH_r\)}
\newcommand{\LnHtParSymbol}{\(LH_r\)}
\newcommand{\GlyphKrnParSymbol}{\(GK_r\)}
\newcommand{\GlyphSzMinParSymbol}{\(GSM_r\)}
\newcommand{\GlyphSzMaxParSymbol}{\(GSX_r\)}
\newcommand{\DnsSampleParSymbol}{\(DS_r\)}
\newcommand{\GlyphRotParSymbol}{\(GRS_r\)}
\newcommand{\PathGenSclParSymbol}{\(PS_r\)}


\section{Parameters}
\label{apx:userparams}
The process of developing mosaics necessitates user manipulation of the following parameters to ensure correct output:
\begin{itemize}
  \paritem
  {Font}
  {The specific glyph set used when stamping glyphs in the rendered image.  }
  {System-installed fonts}
  {\FontParSymbol}

  \paritem
  {Source Image}
  {Used to calculate pixel sizing.}
  {Bitmaps}
  {\SrcImgParSymbol}

  \paritem
  {Source Text}
  {Used to determine which glyph is placed at the successive locations.}
  {UTF-8 String}
  {\SrcTxtParSymbol}

  \paritem
  {Output Scale}
  {The relationship in size of the output image to \SrcImgParSymbol.}
  {ℚ \(\geq 0\)}
  {\OutSclParSymbol}

  \pbodyitem
  {Regions}
  {
    \label{apx:regparams}
    A region most accurately represents a set of pixels which are treated the same when it comes time to render glyphs to paths.
    This is done so that distinct line areas can be treated separately:  If a user wants a text path in a certain section of the image,

    Each region maintains a set of additional parameters which determines the rendered glyphs:
    \begin{itemize}
      \paritem
      {Mask}
      {A 1-channel bitmap representing the area for which glyph points will be considered valid.
        This can be thought of as the effective area of the region.}
      {1-Bit Matrix Congruent to Source Image}
      {\RegMskParSymbol}
      
      \paritem
      {Glyph Path Kernel}
      {A 1-channel bitmap representing the seed of the dilated layers which are then used to generate the region text path.}
      {1-Bit Matrix Congruent to Source Image}
      {\GlyphPathKernelParSymbol}

      \paritem
      {Gutter Line Height}
      {The distance between the first concentric path around the kernel and the kernel itself.
        Zero means the kernel itself will represent the first line.}
      {ℕ}
      {\GtrHtParSymbol}

      \paritem
      {Line Height}
      {The distance between the lines of text on which glyphs are placed.}
      {ℕ}
      {\LnHtParSymbol}

      \paritem
      {Glyph Kerning}
      {The fixed distance between each glyph.}
      {ℕ}
      {\GlyphKrnParSymbol}

      \paritem
      {Glyph Minimum Size}
      {The smallest size glyphs are scaled to.
        This corresponds to the size a glyph would be generated from by encountering a white pixel (lowest density) at its point.}
      {ℕ}
      {\GlyphSzMinParSymbol}

      \paritem
      {Glyph Maximum Size}
      {The largest size glyphs are scaled to.
        This corresponds to the size a glyph would be generated from by encountering a black pixel at its point.
      }
      {ℕ}
      {\GlyphSzMaxParSymbol}

      \paritem
      {Density Sample Size}
      {How many pixels are sampled when a pixel density for a glyph is calculated.}
      {ℕ}
      {\DnsSampleParSymbol}

      \paritem
      {Glyph Rotation Sample Size}
      {How many neighboring pixels on the glyph path are used to calculate the rotation of a given pixel.}
      {ℕ}
      {\GlyphRotParSymbol}

      \paritem
      {Path Generation Scale}
      {By how many times should the matrices that are involved in path generation be scaled up in order to generate text paths.
        Gives higher precision, and therefore resolution, in the resulting path.}
      {ℕ}
      {\PathGenSclParSymbol}

    \end{itemize}
  }
\end{itemize}