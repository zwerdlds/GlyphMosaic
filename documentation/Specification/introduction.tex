% Introduction
\section{Introduction}
GlyphMosaic is a graphic design program.
The application facilitates a specialized graphic design workflow, in which a user supplies a source image, text, and other parameters.
The application then produces a reproduction of the source image using a mosaic of textual elements from a user-supplied source.


\section{System Overview}
\subsection{Scope}
GlyphMosaic is a specialized graphic design application.
It comprises the following functionality:
\begin{itemize}
  \item Enable the user to finely specify the parameters with which the output image is generated.
  \item Produce the output image.
\end{itemize}


\subsection{Context}
The application is installed on a host system.
Users interact with the system's GUI to create, load, and modify graphical mosaics.

\sidiagram
{diagrams/context-diagram.pdf}
{A hypothetical local install of GlyphMosaic and related components.}
{fig:contextDiagram}
{.45}


\subsection{Audience}
This document is intended for developers, open-source contributors, project managers, and consumers involved in the continuous development, maintenance, and use of a GlyphMosaic installation. 
It will provide the audience with a detailed understanding of the underlying architecture of the GlyphMosaic software application to gain better insight into how all components interact to deliver functional and robust productivity.
By reading this document developers and open-source contributors can make informed decisions about how to design and implement various functionalities that will deliver an optimal user experience.
This document aims to provide the audience knowledge to support decisions about identifying potential issues or areas of improvement in the system architecture.
The purpose is to be a comprehensive overview of the GlyphMosaic software architecture to support all audience members involved in the development, maintenance, or usage of a GlyphMosaic application.



\subsection{Statement of Purpose}
GlyphMosaic is thoroughly described in this architectural document.
Diagrams are also supplied to make it easier to comprehend the internal workings of the system.
Various aspects of the GlyphMosaic architecture,  including its core elements, modules, and code structure, are described.
Functionality and quality attributes are described.
The investigation has shown that the GlyphMosaic architecture offers several advantages, including a modular design, flexibility, and usability that have helped it become so popular.
Nevertheless, the investigation has also pointed out other flaws that may be strengthened to improve the application's overall performance, such as its heavy reliance on plugins and the possible security risks they provide.
This document should act as a reference for any person interested in better understanding the GlyphMosaic architecture.

This report attempts to accomplish the following goals:
\begin{itemize}
  \item Describe the complete architecture of the system.
  \item Describe available process alternative methods and their trade-offs.
  \item Provide a broad statement of use for the system itself.
\end{itemize}



\section{Glossary}
\begin{itemize}
  \pbodyitem{Bitmap}{ }
  \pbodyitem{Document}{ }
  \pbodyitem{Glyph}{ }
  \pbodyitem{Graphical User Interface}{ }
  \pbodyitem{Output Image}{ }
  \pbodyitem{Source Image}{ }
  \pbodyitem{Source Text}{ }
  \pbodyitem{Region}{ }
  \pbodyitem{Region Line Kernel}{ }
\end{itemize}



\section{Stakeholders}
\subsection{Stakeholders}
\begin{itemize}
  \pbodyitem{Output Image consumers}{Individuals who view or request graphic design produced by the system.}
  \pbodyitem{Developers}{Individuals responsible for producing the system as described in this document.}
  \pbodyitem{Graphic Designers}{Individuals who produce graphic designs using the system.}
\end{itemize}