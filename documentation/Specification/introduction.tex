\section{Introduction}
GlyphMosaic is a graphic design program.
The application facilitates a specialized graphic design workflow, in which a user supplies a source image, text, and other parameters.
The application then produces a reproduction of the source image using a mosaic of textual elements from a user-supplied source.

This document aims to describe the complete requirements, components, data, algorithms, design, and additional auxiliary details that implement the GlyphMosaic system.


\subsection{Audience}
This document is intended for contributors and consumers involved in the development, maintenance, and use of the GlyphMosaic application.
It aims to provide audiences with a complete description of the underlying organization of the application, facilitating insight into how all system components interact to deliver system functions.

This document's purpose is to provide a comprehensive overview of the GlyphMosaic software architecture to support all audience members involved.
With this document:
\begin{itemize}
    \item Users can understand how the program uses the data they provide.
    \item Analysts can document system requirements.
    \item Designers can document system components and their relations.
    \item Implementers can make informed decisions about how the system should be built.
\end{itemize}


\subsection{Statement of Purpose}
GlyphMosaic is described in this architectural document.
Diagrams are also supplied to facilitate comprehension of the system internals.
Aspects of the GlyphMosaic including context, modules, and code structure are described.
End-user-facing functionality is also briefly described.
This document also highlights architectural compromises and alternatives that were considered during the design phases of the system.
This document should act as a reference for any person interested in better understanding the GlyphMosaic architecture.

This report attempts to accomplish the following goals:
\begin{itemize}
    \item Describe the complete architecture of the system.
    \item Describe available process alternative methods and their trade-offs.
    \item Provide a broad statement of use for the implemented system.
    \item Provide a plan for the implementation of the system.
\end{itemize}