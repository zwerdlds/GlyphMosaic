\section{Architecture}
\subsection{Drivers}
\begin{itemize}
  \lbodyitem{Design Purposes}{
    \pbodyitem{Modeling a Specialized Workflow}{
      The system is designed to facilitate a specific graphic design workflow.
      This workflow follows an interative approach, like most graphic design workflows, in which a document would be created and modified in successive phases until the designer is content with the result.
      At this point, the designer indicates to the system that the image should be rendered at high-resolution for output and consumption by other systems.

      \sidiagram{diagrams/workflow.pdf}
      {The system workflow models a specialized graphic design process.}
      {fig:workflow}
      {\diagsize}
    }
  }
  \lbodyitem{Constraints}{
    \pbodyitem{Resource Use Limitations}{
      Devices serving GlyphMosaic may be constrained by their ability to perform the large number of computations necessary for the production of the end results.
      In particular, resource use may be constrained in the following aspects of the system's functions:
  
      \begin{itemize}
        \pbodyitem{Path Calculation}{
          The system must calculate the path on which glyphs are then vectorized.
          This involves using the region kernel specified by the user and applying a dilation function onto it.
          Over successive operations, this eventually will cover the entire region.
          The accretion operation can be implemented using various methods.
          However, the highest-quality version involves creating an image kernel sized corresponding to the user-defined line distance and applying the transform over a large area.

          The system implements this calculation as detailed in \prettyref{sec:path_calculation}.
        }
        \pbodyitem{Glyph Data}{
          The position and direction of each individual glyph must then be calculated based on the calculated path.
          This involves walking the path and generating locations at user-specified intervals.
          Additionally, each glyph must calculate its scale and color by sampling the source image at the calculated location.
          This operation is computationally intensive due to the large number of glyphs, and must be optimized to this document's performance standards.

          The system implements this calculation as detailed in \prettyref{sec:glyph_data_calculation}.
        }
        \pbodyitem{Glyph Rendering}{
          Each glyph is stamped on a bitmap as part of the rendering process.
          This is constrained by the host system's ability to process potentially thousands of glyphs.
          Glyphs can be stamped on arbitrary bitmaps which can then be merged to form the final rendered bitmap.
          This pattern allows the system to trade memory (additional bitmaps) and processing power (additional glyph-data-consuming threads) for wall-clock time.

          The system implements this calculation as detailed in \prettyref{sec:glyph_render_calculation}.
        }
      \end{itemize}
    }
  }
\end{itemize}


\subsection{Patterns}
The application is implemented at the top level as a unified message bus.
This enables decoupling between other components of the system.
Typed messages are freely submitted to this bus by any component or sub-component of the system.
A shared handler is used to reference the bus by other subsystems.
Subsystems do not send messages directly to one another.


\textbf{Monolithic}
GlyphMosaic communicates with the host operating system using standard methods.
Within the host operating system, the process exists within a singular executable.
This simplifies potential complexity of interacting with the host operating system by reducing the abstract footprint of the application.
In summary: the system need only manage a single executable element, eliminating the need to implement IPC-, or networking-related requirements.
This decision does impact potential performance, as discussed in \prettyref{sec:alt_dist_comp}.


\textbf{Layered}
Within the monolithic system, subsystems compose into macroscopic functionality as layers.
This method of design is an attempt to mitigate complexity of the system by reducing the possible interaction between sub-systems.
This approach is utilized in the following locations:
\begin{itemize}
  \pbodyitem{Test}{Desctiption.}
\end{itemize}


\textbf{Pipe/Filter}
In many cases, systems are composed in series to build larger functionality.
This approach is utilized in the following locations:
\begin{itemize}
  \pbodyitem{Test}{Desctiption.}
\end{itemize}


\textbf{Model-View-Controller}
This approach is utilized in the following locations:
\begin{itemize}
  \pbodyitem{Test}{Desctiption.}
\end{itemize}


\textbf{Event Bus}

This approach is utilized in the following locations:
\begin{itemize}
  \pbodyitem{Test}{Desctiption.}
\end{itemize}


\subsection{Rationales}
A few of the most enabled qualities include:

\begin{itemize}
  \pbodyitem{Performance}{}
  \pbodyitem{Maintainability}{}
\end{itemize}

\subsection{Alternative Architectures}

\begin{itemize}
  \pbodyitem{Distributed Computing}{
    \label{sec:alt_dist_comp}
    Breaking up components of GlyphMosaic into services could enable more computational resources to be added to the system.
    In the future, it may be beneficial to implement a networked architecture to enable the system to scale horizontally and improve responsiveness.
    A distributed approach will need to consider the following aspects:
    \begin{itemize}
      \pbodyitem{File Transfers}{
        Preview and rendered images may be large.
        Operating on bitmaps in memory is substantially faster than over a network.
        Performance profile comparisons between implementations will need to show a clear benefit to this approach before full integration.
      }
      \pbodyitem{Complexity}{
        Interacting with remote compute nodes introduces complexity to the system which is not related to the core functionality of the system.
        Adding, removing, and sending data will all need distinct components to provide a hygenic solution, which will require additional engineering effort.
      }
    \end{itemize}
  }
\end{itemize}


\subsection{Challenges and Limitations}
\subsubsection{Performance}
The system requires computational power to produce result images.
Specifically, the large number of individual glyphs represent a potential scaling issue.
In addition, the large bitmaps the system may interact with may also contribute to performance degradation.
