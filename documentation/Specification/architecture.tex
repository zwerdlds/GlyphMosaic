\section{Architecture}
\subsection{Drivers}
\begin{itemize}
  \lbodyitem{Design Purposes}{
    \pbodyitem{Modeling a Specialized Workflow}{
      The system is designed to facilitate a specific graphic design workflow.
      This workflow follows an interative approach, like most graphic design workflows, in which a document would be created and modified in successive phases until the designer is content with the result.
      At this point, the designer indicates to the system that the image should be rendered at high-resolution for output and consumption by other systems.

      \lidiagram{diagrams/workflow.pdf}
      {The system workflow includes specialized phases.}
      {fig:workflow}
    }
  }
  \pbodyitem{Use Cases}{See \prettyref{sec:usecases}}
  \pbodyitem{Quality Attributes}{See \prettyref{sec:qualityattributes}}
  \lbodyitem{Constraints}{
    \pbodyitem{Resource Use Limitations}{
      Devices serving GlyphMosaic may be constrained by processing power and memory.
    }
  }
\end{itemize}


\subsection{Styles and Patterns}
GlyphMosaic communicates with the host operating system using standard methods.

\textbf{Monolithic}
Within the host operating system, the process exists within a singular executable.
This simplifies potential complexity 


\textbf{Layered}
Within the monolithic system, subsystems compose into macroscopic functionality as layers.
This method of design is an attempt to mitigate complexity of the system.
This approach is utilized in the following locations:
\begin{itemize}
  \pbodyitem{Test}{Desctiption.}
\end{itemize}


\textbf{Pipe/Filter}
In many cases, systems are composed in series to build larger functionality.
This approach is utilized in the following locations:
\begin{itemize}
  \pbodyitem{Test}{Desctiption.}
\end{itemize}


\textbf{Model-View-Controller}
This approach is utilized in the following locations:
\begin{itemize}
  \pbodyitem{Test}{Desctiption.}
\end{itemize}


\textbf{Event Bus}

This approach is utilized in the following locations:
\begin{itemize}
  \pbodyitem{Test}{Desctiption.}
\end{itemize}


\subsection{Rationales}
A few of the most enabled qualities include:

\begin{itemize}
  \pbodyitem{Performance}{}
  \pbodyitem{Maintainability}{}
\end{itemize}

\subsection{Alternative Architectures}

\begin{itemize}
  \pbodyitem{Distributed Computing}{
    Breaking up components of GlyphMosaic into services could enable more computational resources to be added to the system.
    % However, it is still the case in this implementation that updates must be done all at once: the system is still unitary.

    % The typical use of load-balancers and in-memory caches which large-scale deployments employ may be interpreted as an application of this approach.

    % Narrowing the focus to the GlyphMosaic system, there are two main reasons why it may not be preferential to implement this approach:
    % \begin{itemize}
    %   \pbodyitem{Complexity}{
    %     Breaking up the monolithic internal structure of the system into components would require substantial refactoring.
    %     Plug-Ins, represent a substantial barrier: Hooks must be registered centrally, so would need to be resolved somehow.
    %     Without involving complex voting protocols, this alone would still represent a single point of communication.
    %     Additionally, the state of the system can be queried at the time that a hook is activated, demanding complete synchronization between all nodes.

    %     It may be possible for templates to be distributed since their outputs should be idempotent.
    %     In practice, however, since they are capable of executing arbitrary PHP, they would probably encounter similar issues to plug-ins.
    %   }

    %   \pbodyitem{Diminishing Returns of Scale}{
    %     While performance on GlyphMosaic may not be superior to other platforms, most users do not encounter traffic sufficient to demand the scalability provided by SOA or microservice architectures.
    %   }
    % \end{itemize}

    % In conclusion, the monolithic model is ``good enough'' compared to a service-oriented architecture.
    % The main reason this approach is unlikely to have advantages for system stakeholders is the diminishing returns to scale those stakeholders will perceive.
  }
\end{itemize}


\subsection{Challenges and Limitations}
% A three-tier architecture provides several benefits to the GlyphMosaic system, but it also comes with some limitations that can affect certain quality attributes.
% Developers and organizations need to consider these limitations when designing and implementing their systems to ensure they meet their requirements and objectives.
% Proper planning, design, and testing can help mitigate these limitations and ensure a successful implementation.
% A few drawbacks to consider include:
% \begin{itemize}
% \pbodyitem{Performance}{
%   Performance limitations can occur due to the communication overhead between the layers.
%   Network latency and message passing can add to the overall response time of the system, leading to slower performance.
% }
% \end{itemize}
\subsubsection{Performance}
