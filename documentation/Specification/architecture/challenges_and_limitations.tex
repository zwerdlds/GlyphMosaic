\subsection{Challenges and Limitations}
\subsubsection{Performance}
The system requires computational power to produce result images.
Specifically, the large number of individual glyphs represents a potential scaling issue.
In addition, the large bitmaps the system may interact with may also contribute to performance degradation.

This should be mitigated via the following techniques:
\begin{itemize}
    \item Limiting memory allocations where possible via flyweights.
    \item Limiting memory reallocations where possible via heap pools.
    \item Using data structures with limited size where possible.
    \item Arrays and vectors should be reallocated or resized as little as possible.
    \item {
          When the \index{Parameters!Source Text} changes, dependant vectors should be pre-allocated and preserved until it is changed further.
          This implementation detail is accepted -- there may be memory use considerations to implement this trade for performance.
          }
\end{itemize}


\subsubsection{Malleability}
The system should be designed in such a way that design changes are easily executed.
To enforce this purpose, best practices should be followed to implement a decoupled design.
In practice, these principles are embodied by ``SOLID'' principles.

The project should implement them by using the following guidelines:
\begin{itemize}
    \item {
          Use commonly acknowledged OOP-style patterns to implement behavior.
          This will be performed by using the ROOPES library abstractions as often as possible.
          }
    \item {
          Limit internal components to a single responsibility.
          Since a majority of abstract behavior should be implemented by ROOPES, most internal components should have minimal implementations.

          In the inevitable cases where custom behavior is necessary, the implementation should be encapsulated as much as possible, and implement the appropriate ROOPES primitive abstraction traits (Executable, Emitter, Handler, Transformer).
          Where possible, explicit and generic types should be used to enforce correctness across component boundaries.
          }
    \item {
          Inter-module dependencies should implement behavioral dependencies via traits, not structs.
          This may cause some dependencies to require \texttt{Box<dyn ... >} types that demand heap allocations.
          This implementation detail is accepted -- there may be performance degradation associated with these allocations.
          This should be balanced by pre-allocating as many resources as possible singly and not re-performed, either by using heap pools, flyweights, and pre-allocations.
          }
    \item {
          Allow components to inject internal behavior into changes of data from other components.
          At the subsystem level, this could be accomplished by defining a publisher to which other subsystems can subscribe to events.

          For example, as parameters are changed, the UI subsystems may modify attributes defined in the Document subsystem, via the facade-provided publisher.
          During system launch, the Persistence subsystem would have provided a subscriber to that publisher which would then be notified of changes.
          Behavior internal to the Persistence subsystem would be performed as appropriate.
          In this way, behavior local to Persistence can be triggered by UI without direct dependencies, decoupling the two subsystems.
          }
\end{itemize}