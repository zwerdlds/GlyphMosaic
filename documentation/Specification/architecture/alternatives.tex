\subsection{Alternative Architectures}
\begin{itemize}
  \pbodyitem{Distributed Computing}{
    \label{sec:alt_dist_comp}
    Breaking up components of GlyphMosaic into services could enable more computational resources to be added to the system.
    In the future, it may be beneficial to implement a networked architecture to enable the system to scale horizontally and improve responsiveness.
    A distributed approach will need to consider the following aspects:
    \begin{itemize}
      \pbodyitem{File Transfers}{
        Preview and rendered images may be large.
        Operating on bitmaps in memory is substantially faster than over a network.
        Performance profile comparisons between implementations will need to show a clear benefit to this approach before full integration.
      }
      \pbodyitem{Complexity}{
        Interacting with remote compute nodes introduces complexity to the system which is not related to the core functionality of the system.
        Adding, removing, and sending data will all need distinct components to provide a hygienic solution, which will require additional engineering effort.
      }
    \end{itemize}
  }
  \pbodyitem{Shader-based Modules}{
    \label{sec:shaders}
    An intermediate enhancement to the system would be to replace many of the bitmap-level transformations with shaders.
    This approach would have architectural impacts on the system but would enable much faster feedback by enabling GPU matrix transformations.
    This implementation will be reconsidered in the future.
  }
  \pbodyitem{AI Platforms}{
    \label{sec:ai_plat}
    Generative AI neural networks present an interesting application to the process of isolating regions and creating text paths, which could have a substantial impact on the architecture of the system.
    At this time, however, creating training these models would be prohibitively expensive, and so are not considered in the current form of the design.
  }
\end{itemize}