\subsection{Alternative Architectures}
\begin{itemize}
  \pbodyitem{Distributed Computing}{
    \label{sec:alt_dist_comp}
    Breaking up components of GlyphMosaic into services, to be run on separate hosts could enable more computational resources to be dynamically added to the system.
    It may be beneficial to implement a networked architecture to enable the system to scale horizontally so that the local system does not represent a computational constraint, and improve responsiveness.
    A distributed approach will need to consider the following aspects:
    \begin{itemize}
      \pbodyitem{File Transfers}{
        Preview and rendered images may be large.
        Operating on bitmaps in memory is substantially faster than over a network.
        Performance profile comparisons between implementations will need to show a clear benefit to this approach before full integration.
      }
      \pbodyitem{Complexity}{
        Interacting with remote compute nodes introduces complexity to the system which is not related to its core functionality.
        Adding, removing, and sending data will all need distinct components to provide a hygienic solution, which will require additional engineering effort.
      }
    \end{itemize}
  }
  \pbodyitem{Shaders}{
    \label{sec:shaders}
    An intermediate enhancement to the system would be to replace many of the CPU-based transformations with shaders.
    This approach may have architectural impacts, but would enable much faster feedback by enabling matrix transformations on the GPU.
    This implementation will be considered in the future if the performance criteria are not being met.
  }
  \pbodyitem{AI Platforms}{
    \label{sec:ai_plat}
    Generative AI neural networks present an interesting application to the process of isolating regions, and glyph path generation.
    Implementing this approach could replace a substantial amount of the existing design, but it could also eliminate a substantial amount of the work necessary on the part of users.
    This approach is not considered in the current form of the design, but it may be possible in the future to create training data from 3-D renders, merged with a midjourney-like model.
    This approach would also need to carefully consider the problem of compatibility, since video card interfaces such as ROCm or CUDA are both common candidate platforms.
  }
\end{itemize}