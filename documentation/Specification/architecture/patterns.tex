\subsection{Patterns}
The application is implemented at the top level as a unified message bus.
This enables decoupling between other components of the system.
Typed messages are freely submitted to this bus by any component or sub-component of the system.
A shared handler is used to reference the bus by other subsystems.
Subsystems do not send messages directly to one another.

\begin{itemize}
  \pbodyitem{Monolithic}{
    GlyphMosaic communicates with the host operating system using standard methods.
    Within the host operating system, the process exists within a singular executable.
    This simplifies potential complexity of interacting with the host operating system by reducing the abstract footprint of the application.
    In summary: the system need only manage a single executable element, eliminating the need to implement IPC-, or networking-related requirements.
    This decision does impact potential performance, as discussed in \prettyref{sec:alt_dist_comp}.
  }


  \pbodyitem{Layered}{
    Within the monolithic system, subsystems compose into macroscopic functionality as layers.
    This method of design is an attempt to mitigate complexity of the system by reducing the possible interaction between sub-systems.
    This approach is utilized in the following locations:
    \begin{itemize}
      \pbodyitem{Test}{Desctiption.}
    \end{itemize}
  }


  \pbodyitem{Pipe/Filter}{
    In many cases, systems are composed in series to build larger functionality.
    This approach is utilized in the following locations:
    \begin{itemize}
      \pbodyitem{Test}{Desctiption.}
    \end{itemize}
  }


  \pbodyitem{Model-View-Controller}{
    This approach is utilized in the following locations:
    \begin{itemize}
      \pbodyitem{Test}{Desctiption.}
    \end{itemize}
  }


  \pbodyitem{Event Bus}{
    This approach is utilized in the following locations:
    \begin{itemize}
      \pbodyitem{Test}{Desctiption.}
    \end{itemize}
  }
\end{itemize}