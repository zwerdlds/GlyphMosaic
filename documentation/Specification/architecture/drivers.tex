\subsection{Drivers}
\begin{itemize}
  \lbodyitem{Design Purposes}{
    \pbodyitem{Modeling a Specialized Workflow}{
      \label{sec:user_workflow_model}
      The system is designed to facilitate a specific graphic design workflow.
      This workflow follows an iterative approach, similar to most graphic design workflows, in which a document is created and modified in successive phases until the designer is content with the result.
      Eventually, the designer indicates to the system that the image should be rendered at high resolution for output and consumption by other systems.

      \sidiagram{diagrams/workflow.pdf}
      {The system workflow models a specialized graphic design process.}
      {fig:workflow}
      {\diagsize}
    }
  }
  \lbodyitem{Constraints}{
    \pbodyitem{Resource Use Limitations}{
      Devices serving GlyphMosaic may be constrained by their ability to perform a large number of computations necessary for the production of results.
      These implement the algorithm detailed in  \prettyref{apx:algorithm}.
      In particular, resource use may be constrained in the following aspects of the system's functions:

      \begin{itemize}
        \pbodyitem{Path Calculation}{
          The system must calculate the path on which glyphs are then vectorized.
          This involves using the region kernel specified by the user and applying a dilation function to it.
          Over successive operations, this eventually will cover the entire region.
          The accretion operation can be implemented using various methods.
          However, the highest-quality version involves creating an image kernel-sized corresponding to the user-defined line distance and applying the transform over a large area.
        }
        \pbodyitem{Glyph Data}{
          The position and direction of each glyph must then be calculated based on the calculated path.
          This involves walking the path and generating locations at user-specified intervals.
          Additionally, each glyph must calculate its scale and color by sampling the source image at the calculated location.
          This operation is computationally intensive due to the large number of glyphs and must be optimized to this document's performance standards.
        }
        \pbodyitem{Glyph Rendering}{
          Each glyph is stamped on a bitmap as part of the rendering process.
          This is constrained by the host system's ability to process potentially thousands of glyphs.
          Glyphs can be stamped on arbitrary bitmaps which can then be merged to form the final rendered bitmap.
          This pattern allows the system to trade memory (additional bitmaps) and processing power (additional glyph-data-consuming threads) for wall-clock time.
        }
      \end{itemize}
    }
  }
\end{itemize}