\begin{appendices}
  \section{Per-Region Parameters}
  \label{apx:userparams}
  A user has the ability to modify various parameters of a document:
  \begin{itemize}
    \pbodyitem{Font}{
      The specific glyph set used when stamping glyphs in the rendered image.
    }
    \pbodyitem{Source Image}{
      Used to calculate pixel sizing.
    }
    \pbodyitem{Source Text}{
      Used to determine which glyph is placed at the successive locations.
    }
    \pbodyitem{Regions}{
      When a region is added, it is `empty' in that it will not allow any glyph points to be added to the output image.
      Removing a region is a cascading delete - all parameters associated with the region are also removed.
      \begin{itemize}
        \pbodyitem{Mask}{
          A 1-channel bitmap representing the area for which glyph points will be considered valid.
          This can be thought of as the effective area of the region.
        }
        \pbodyitem{Glyph Path Kernel}{
          A 1-channel bitmap representing the seed of the dilated layers which are then used to generate the region text path.
        }
        \pbodyitem{Line Height}{
          The distance between the lines of text on which glyphs are placed.
        }
        \pbodyitem{Glyph Kerning}{
          The fixed distance between each glyph.
        }
        \pbodyitem{Glyph Minimum Size}{
          The smallest size glyphs are scaled to.
          This corresponds to the size a glyph would be generated from by encountering a white pixel at its point.
        }
        \pbodyitem{Glyph Maximum Size}{
          The largest size glyphs are scaled to.
          This corresponds to the size a glyph would be generated from by encountering a black pixel at its point.
        }
      \end{itemize}
    }
  \end{itemize}
\end{appendices}