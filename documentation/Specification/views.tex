\section{Architectural Views}
\label{sec:views}
\subsection{Top-Level System}
The system operates as a single abstract running process, either in a browser or within the host operating system directly.

\sidiagram
{diagrams/app_host}
{The GlyphMosaic system has limited responsibility from the host system perspective.}
{fig:cnc}
{\diagsize}

The system interacts with the following aspects of the host system:
\begin{itemize}
  \pbodyitem{Filesystem}{
    \begin{itemize}
      \item Source Image is loaded from the host filesystem.
      \item Source Text is loaded from the host filesystem.
      \item \Glspl{out_image} are written to the host filesystem.
      \item \Glspl{document} are read and written to the host filesystem.
    \end{itemize}
  }
  \pbodyitem{Window Manager}{
    For appropriate targets, the application presents a window-based GUI for designing and rendering Output Images.
  }
  \pbodyitem{Web Browser}{
    For appropriate targets, the application presents a window-based GUI for designing and rendering Output Images.
  }
\end{itemize}


\subsection{Modules}
\label{sec:modules}
See \prettyref{sec:subsystems} for a complete listing of top-level modules.


\subsubsection{Module Dependencies}
Application subsystems must rely on each other to enable the program's function.
\prettyref{fig:modules_top} shows the module dependency hierarchy.
Below, each relationship is explained.
See \prettyref{sec:subsystems} for further details on the dependency relationships.

\docmod, \uimod, \permod, and \bmmod\ are omitted from the discussion here because those modules use or are used by the subsystems listed here.
In effect, the modules listed here include those central to the calculation of the program.

The relationships described here refer to objects set up in the booting of the application, being defined by a \sysfact\ instance via \sysbldr\ which is aware of all module's facades.
\sysbldr\ is responsible for wiring up subscribers with associated publishers, as described below.
As a result, these relationships are not necessarily typed, they are set up by the \sysfact.

\sidiagram{diagrams/top_level_modules}
{The top-level dependencies of the project's subsystems.}
{fig:modules_top}
{\diagsize}

\begin{enumerate}
  \item \linemod\ depends on \regmod\ to be notified when a Region Mask changes.
  \item \ptmod\ depends on \linemod\ to be notified when a region's text path changes.  This path affects the location of the region's points as well as their tangents.
  \item \ptmod\ depends on \regmod\ to be notified when the Region Mask changes.  This \bmtype\ affects which region points are retained.
  \item \glymod\ depends on \ptmod\ to be notified of changes to point locations and tangents.  These values are used to place and rotate individual glyphs.
\end{enumerate}

\nosplit{
  \subsection{Data Flows}
  \label{sec:dataflows}
  Data flow between each module in such a way as to accomplish the workflow outlined in \prettyref{apx:algorithm}.
  This process somewhat mirrors the components described in \prettyref{sec:modules}, but also describes the relationships necessary to update the user interface and document.

  Specifically, when parameters are changed, they trigger recalculations of downstream data, according to the relationships described here:
  \begin{itemize}
    \item source image: Glyphs \\
          The source image defines the scale of glyphs, and changing it has the likely impact of changing the result of the line point's sample pixels.
          This means that changing the source image should cause glyphs to be recalculated.

    \item source text: Glyphs


    \item font: Output Image

    \item output scale: Output Image

    \item region mask: Points

    \item region glyph path kernel: Lines

    \item region gutter line height: Lines

    \item region line height: Lines

    \item region glyph kerning: Points

    \item region glyph minimum size: Glyphs

    \item region glyph maximum size: Glyphs

    \item region density sample size: Glyphs

    \item region glyph rotation sample size: Points

    \item region path generation scale: Lines

  \end{itemize}

  In addition, when subsystems' data are changed, they also trigger their own downstream recalculations:

  \lidiagram{diagrams/data_flow_subsys}
  {Changes to document parameters will have cascading effects on downstream subsystems.}
  {fig:data_flow}

  \begin{itemize}
    \item Lines triggers Points \\
          Changes to a line will likely change the location of points.
          Lines define the locations the point walking algorithm will select from.

    \item
  \end{itemize}

}
