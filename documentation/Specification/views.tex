\section{Architectural Views}{
  \label{sec:views}
  \nosplit{
    \subsection{Top-Level System}
    The system interacts with the following aspects of the host system:
    \begin{itemize}
      \pbodyitem{Filesystem}{
        The system reads source images and text from the host filesystem.
        \Glspl{out_image} are written to the host filesystem.
        \Glspl{document} are read and written to the host filesystem.
      }
    \end{itemize}

    \sidiagram
    {diagrams/app_host}
    {The GlyphMosaic system has limited responsibility from the host system perspective.}
    {fig:cnc}
    {\diagsize}
  }
  \subsection{Modules}
  \label{sec:modules}
  The system is composed of the following top-level modules:
  \begin{itemize}
    \pbodyitem{UI}{
      The user interface manages the frontend of the system.
      In particular, it manages the state of the UI.

      The UI module publishes the following data:
      \begin{itemize}
        \pbodyitem{Region Parameters}{
          When a region parameter is updated by a user, the UI module is responsible for forwarding that change to the remainder of the system.
        }
        \pbodyitem{Direct \gls{bitmap} Manipulation}{
          The user may draw on the canvas to modify \glspl{path_dilation_mask} or \glspl{glyph_path_kernel}.
        }
        \pbodyitem{File Save}{
          The user may specify a location from which a \gls{document} or \gls{out_image} can be saved.
        }
        \pbodyitem{File Load}{
          The user may specify a location from which a \gls{document}, \gls{src_img}, or \gls{src_txt} can be loaded.
        }
      \end{itemize}
      The UI must also reflect when a relevant change is made to the internal system, to allow the user to view that change.
      These changes are also forwarded to the document management system to ensure the change is reflected when the document is exported.
    }
  \end{itemize}

  \subsubsection{Module Dependencies}
  Application subsystems must rely on each other to enable the program's function.
  \prettyref{fig:modules_top} shows the module dependency hierarchy.
  Below, each relationship is explained.
  See \prettyref{sec:subsystems} for further details in the dependency relationships.

  \sidiagram{diagrams/top_level_modules}
  {The top-level dependencies of the project's subsystems.}
  {fig:modules_top}
  {\diagsize}

  \begin{enumerate}
    \item \linemod\ depends on \regmod\ to be notified when a Region Mask changes.
    \item \linemod\ depends on \bmmod:
          \begin{itemize}
            \item The Region Mask is a \bmtype\ which is queried to determine the total number of concentric dilations to perform.
            \item The Glyph Path Kernel is a \bmtype\ which is dilated to form each concentric line.
            \item Dilation stages of the Glyph Path are generated in \bmtype\ s before being translated to a type supplied by lyon.
          \end{itemize}
    \item \ptmod\ depends on \linemod\ to be notified when a region's text path changes.  This path affects the location of the region's points as well as their tangents.
    \item \ptmod\ depends on \regmod\ to be notified when the Region Mask changes.  This \bmtype\ affects which region points are retained.
    \item \ptmod\ depends on \bmmod:
          \begin{itemize}
            \item The Region Mask is a \bmtype\ which is queried to determine which points are retained.
            \item The Source Image is a \bmtype\ which is queried to determine the color and intensity attributes of the point.
          \end{itemize}
    \item \ptmod\ depends on \srcmod\ to be notified when the Source Image changes.  This \bmtype\ is sampled to determine the size and color values of points.
    \item \glymod\ depends on \ptmod\ to be notified of changes to point locations and tangents.  These values are used to place and rotate individual glyphs.
    \item \glymod\ depends on \srcmod\ to be notified of changes to the source text.  This string is then sampled to determine which glyph corresponds to a given point.
    \item \regmod\ depends on \bmmod.  The Region Mask is a \bmtype\ which can be modified and sent to subscribing components.
    \item \uimod\ depends on \docmod\ to save, load, and create files.
    \item \uimod\ depends on \glymod\ to be notified when the glyph preview and final images have changed.
          \uimod\ may also send changes to these values originating from user input.
    \item \uimod\ depends on \ptmod\ to be notified when the point preview image has changed.
          \uimod\ may also send changes to these values originating from user input.
    \item \uimod\ depends on \linemod\ to be notified when the line preview image has changed.
          \uimod\ may also send changes to these values originating from user input.
    \item \uimod\ depends on \regmod\ to be notified when the region preview image has changed.
          \uimod\ may also send changes to these values originating from user input.
    \item \uimod\ depends on \srcmod\ to be notified when the source preview image has changed.
          \uimod\ may also send changes to these values originating from user input.
    \item \uimod\ depends on \bmmod.  All preview images are \bmtype s.
    \item \docmod\ depends on \bmmod to encode and decode stored \bmtype s.
    \item \docmod\ depends on \glymod\ to be notified when any of the glyph input parameters have changed.
    \item \docmod\ depends on \ptmod\ to be notified when any of the point input parameters have changed.
    \item \docmod\ depends on \regmod\ to be notified when any of the region input parameters have changed.
    \item \docmod\ depends on \linemod\ to be notified when any of the line input parameters have changed.
    \item \docmod\ depends on \srcmod\ to be notified when any of the source parameters have changed.
    \item The Source Image is a \bmtype\ which is queried to determine the color and intensity attributes of points.
  \end{enumerate}

  \nosplit{
    \subsection{Data Flows}
    \label{sec:dataflows}
    Data flows between each module in such a way as to accomplish the workflow set forth in \prettyref{apx:algorithm}.
    This process somewhat mirrors the components described in \prettyref{sec:modules}, but also describes the relationships necessary to update the user interface and document.

    \sidiagram{diagrams/data_flow}
    {Succinct description.}
    {fig:data_flow}
    {\diagsize}
  }
 }