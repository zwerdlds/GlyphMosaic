\section{Architectural Views}
\label{sec:views}
\subsection{Top-Level System}
\sidiagram
{diagrams/app_host}
{The GlyphMosaic system has limited responsibility from the host system perspective.}
{fig:cnc}
{\diagsize}

The system interacts with the following aspects of the host system:
\begin{itemize}
  \pbodyitem{Filesystem}{
    \begin{itemize}
      \item Source Image is loaded from the host filesystem.
      \item Source Text is loaded from the host filesystem.
      \item \Glspl{out_image} are written to the host filesystem.
      \item \Glspl{document} are read and written to the host filesystem.
    \end{itemize}
  }
  \pbodyitem{Window Manager}{
    For appropriate targets, the application presents a window-based GUI for designing and rendering Output Images.
  }
  \pbodyitem{Web Browser}{
    For appropriate targets, the application presents a window-based GUI for designing and rendering Output Images.
  }
\end{itemize}


\subsection{Modules}
\label{sec:modules}
See \prettyref{sec:subsystems} for a complete listing of top-level modules.


\subsubsection{Module Dependencies}
Application subsystems must rely on each other to enable the program's function.
\prettyref{fig:modules_top} shows the module dependency hierarchy.
Below, each relationship is explained.
See \prettyref{sec:subsystems} for further details on the dependency relationships.

\docmod, \uimod, \permod, and \bmmod\ are omitted from the discussion here because those modules use or are used by the subsystems listed here.
In effect, the modules listed here include those central to the calculation of the program.

The relationships described here refer to objects set up in the booting of the application, being defined by a \sysfact\ instance via \sysbldr\ which is aware of all module's facades.
\sysbldr\ is responsible for wiring up subscribers with associated publishers, as described below.
As a result, these relationships are not necessarily typed, they are set up by the \sysfact.

\sidiagram{diagrams/top_level_modules}
{The top-level dependencies of the project's subsystems.}
{fig:modules_top}
{\diagsize}

\begin{enumerate}
  \item \linemod\ depends on \regmod\ to be notified when a Region Mask changes.
  \item \ptmod\ depends on \linemod\ to be notified when a region's text path changes.  This path affects the location of the region's points as well as their tangents.
  \item \ptmod\ depends on \regmod\ to be notified when the Region Mask changes.  This \bmtype\ affects which region points are retained.
  \item \ptmod\ depends on \srcmod\ to be notified when the Source Image changes.  This \bmtype\ is sampled to determine the size and color values of points.
  \item \glymod\ depends on \ptmod\ to be notified of changes to point locations and tangents.  These values are used to place and rotate individual glyphs.
  \item \glymod\ depends on \srcmod\ to be notified of changes to the source text.  This string is then sampled to determine which glyph corresponds to a given point.
\end{enumerate}

\nosplit{
  \subsection{Data Flows}
  \label{sec:dataflows}
  Data flow between each module in such a way as to accomplish the workflow outlined in \prettyref{apx:algorithm}.
  This process somewhat mirrors the components described in \prettyref{sec:modules}, but also describes the relationships necessary to update the user interface and document.

  \sidiagram{diagrams/data_flow}
  {Succinct description.}
  {fig:data_flow}
  {\diagsize}
}
