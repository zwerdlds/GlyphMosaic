\section{Architectural Views}{
  \label{sec:views}
  \nosplit{
    \subsection{Top-Level System}
    The system interacts with the following aspects of the host system:
    \begin{itemize}
      \pbodyitem{Filesystem}{
        The system reads source images and text from the host filesystem.
        Output images are written to the host filesystem.
        Application documents are read and written to the host filesystem.
      }
    \end{itemize}

    \sidiagram
    {diagrams/app_host}
    {The GlyphMosaic system has limited responsibility from the host system perspective.}
    {fig:cnc}
    {\diagsize}
  }
  \nosplit{
    \subsection{Modules}
    \label{sec:modules}
    The system is composed of the following top-level modules:
    \begin{itemize}
      \pbodyitem{UI}{
        The user interface manages the frontend of the system.
        In particular, it manages the state of the UI.

        The UI module publishes the following data:
        \begin{itemize}
          \pbodyitem{Region Parameters}{
            When a region parameter is updated by a user, the UI module is responsible for forwarding that change to the remainder of the system.
          }
          \pbodyitem{Bitmap Draw Events}{
            The user may draw on the canvas to modify a region mask or a path line kernel.
          }
          \pbodyitem{File Save}{
            The user may specify a location from which a file must be saved.
          }
          \pbodyitem{File Load}{
            The user may specify a location from which a file must be loaded.
          }
        \end{itemize}
        The UI must also reflect when a relevant change is made to the internal system, to allow the user to view that change.
        These changes are also forwarded to the document management system to ensure the change is reflected when the document is exported.
      }
    \end{itemize}

    \sidiagram{diagrams/top_level_modules}
    {The top-level organization of the system.}
    {fig:modules_top}
    {\diagsize}
  }
  \nosplit{
    \subsection{Data Flows}
    \label{sec:dataflows}
    Data flows between each module in such a way as to accomplish the workflow set forth in \prettyref{sec:user_workflow_model}.
    This process somewhat mirrors the components described in \prettyref{sec:modules}, but also describes the relationships necessary to update the user interface and document.
  
    \sidiagram{diagrams/data_flow}
    {Succinct description.}
    {fig:data_flow}
    {\diagsize}
  }
 }