\section{Architectural Views}
\label{sec:views}
\subsection{Top-Level System}
The system operates as a single abstract running process, either in a browser or within the host operating system directly.

\sidiagram
{diagrams/app_host}
{The GlyphMosaic system has limited responsibility from the host system perspective.}
{fig:cnc}
{\diagsize}

The system interacts with the following aspects of the host system:
\begin{itemize}
      \pbodyitem{Filesystem}{
            \begin{itemize}
                  \item Source Image is loaded from the host filesystem.
                  \item Source Text is loaded from the host filesystem.
                  \item \Glspl{out_image} are written to the host filesystem.
                  \item \Glspl{document} are read and written to the host filesystem.
            \end{itemize}
      }
      \pbodyitem{Window Manager}{
            For appropriate targets, the application presents a window-based GUI for designing and rendering Output Images.
      }
      \pbodyitem{Web Browser}{
            For appropriate targets, the application presents a window-based GUI for designing and rendering Output Images.
      }
\end{itemize}


\subsection{Modules}
\label{sec:modules}
See \prettyref{sec:subsystems} for a complete listing of top-level modules.


\subsubsection{Module Dependencies}
\label{sec:moddep}
Application subsystems must rely on each other to enable the program's function.
\prettyref{fig:modules_top} shows the module dependency hierarchy.
Below, each relationship is explained.
See \prettyref{sec:subsystems} for further details on the dependency relationships.

\docmod, \uimod, \permod, and \bmmod\ are omitted from the discussion here because those modules use or are used by the subsystems listed here.
In effect, the modules listed here include those central to the calculation of the program.

The relationships described here refer to objects set up in the booting of the application, being defined by a \sysfact\ instance via \sysbldr\ which is aware of all module's facades.
\sysbldr\ is responsible for wiring up subscribers with associated publishers, as described below.
As a result, these relationships are not necessarily typed, they are set up by the \sysfact.

\sidiagram{diagrams/top_level_modules}
{The top-level dependencies of the project's subsystems.}
{fig:modules_top}
{\diagsize}

\begin{enumerate}
      \item \linemod\ depends on \regmod\ to be notified when a Region Mask changes.
      \item \ptmod\ depends on \linemod\ to be notified when a region's text path changes.  This path affects the location of the region's points as well as their tangents.
      \item \ptmod\ depends on \regmod\ to be notified when the Region Mask changes.  This \bmtype\ affects which region points are retained.
      \item \lettermod\ depends on \ptmod\ to be notified of changes to point locations and tangents.  These values are used to place and rotate individual Letters.
\end{enumerate}




\subsection{Data Flows}
\label{sec:dataflows}
Data flow between each module in such a way as to accomplish the workflow outlined in \prettyref{sec:moddep} and \prettyref{apx:algorithm}.
This section also describes the relationships necessary to update the user interface and document properties not necessarily exposed directly to the user.

\lidiagram{diagrams/data_flow_subsys}
{Changes to document parameters will have cascading effects on downstream subsystems.}
{fig:data_flow}

Specifically, when parameters are changed, they trigger recalculations of downstream data, according to these relationships:
\begin{itemize}
      \item source image: Letters \\
            The source image defines the scale of letters, and changing it has the likely impact of changing the result of the line point's sample pixels.
            This means that changing the source image must cause letter data to be recalculated.

      \item source text: Letters \\
            The source text is placed on the output image bitmap glyph by glyph, transformed by each element in the line point list.
            This means that changing the text must cause letter data to be recalculated.

      \item font: Output Image \\
            The font face determines the specific glyphs that are used to represent letter data, being placed and scaled according to point data.
            This means changing the font will cause the output image to be recalculated.

      \item output scale: Output Image \\
            The output scale of an image determines the location and size of the glyphs as they are stamped according to point data.
            This means changing the output scale will cause the output image to be recalculated.

      \item region mask: Points, Lines \\
            A region mask represents the area for which a region line's points are valid and therefore is responsible for filtering line points.
            This means changing a region mask must cause points to be recalculated. \\
            \\
            In addition, a region mask is used to calculate the number of lines required for the region to be fully covered in text.
            This means changing a region mask must cause lines to be recalculated.


      \item region glyph path kernel: Lines \\
            A region glyph path kernel is used to iteratively calculate the region's lines using progressively larger circular radii, involving the region line and gutter height.
            This means changing a path kernel must cause the lines to be recalculated.

      \item region gutter line-height: Lines \\
            A region gutter line height affects the offset of a region's lines.
            This means changing a gutter height must cause the lines to be recalculated.

      \item region line-height: Lines \\
            A region line height is used to iteratively calculate the region's lines using progressively larger circular radii.
            This means changing a line height must cause the lines to be recalculated.

      \item region glyph kerning: Points \\
            A region's glyph kerning values affect the inter-point spacing while walking the region line.
            This means that changing kerning must cause the points to be recalculated.

      \item region glyph minimum size: Letters \\
            The glyph minimum size is used in calculating individual letter scales.
            This means that changing the minimum size must cause the letter data to be recalculated.

      \item region glyph maximum size: Letters \\
            The glyph maximum size is used in calculating individual letter scales.
            This means that changing the maximum size must cause the letter data to be recalculated.

      \item region density sample size: Letters \\
            The density sample size is used to determine which pixels are sampled when a point scale is calculated.
            This means that changing the density sample size must cause the letter data to be recalculated.

      \item region glyph rotation sample size: Points \\
            The rotation sample size is used to determine which points are sampled when a point rotation is calculated.
            This means that changing the rotation sample size must cause the letter data to be recalculated.

      \item region path generation scale: Lines \\
            Lines are calculated using a bitmap and edge detection, and the scale of that bitmap can affect the amount of precision provided while calculating lines.
            Changing this value will change the location of the successive lines, and must therefore require them to be recalculated.
\end{itemize}

In addition, when subsystems' data are changed, they also trigger their downstream recalculations:
\begin{itemize}
      \item Lines triggers Points \\
            Changes to a line will likely change the location of points.
            Lines define the locations the point walking algorithm will select from.

      \item Points triggers Letters \\
            Changes to a region's points will cause the parameters associated with each letter to change.
            Points define the location and rotation of each letter.

      \item Letters triggers Output Image \\
            Changes to any letter will cause the output image to change.
            Letters define the location, size, and rotation of glyphs which are stamped to create the output image.
\end{itemize}

